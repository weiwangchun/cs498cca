% !TEX TS-program = pdflatex
% !TEX encoding = UTF-8 Unicode

\documentclass[10pt]{article}
\usepackage[utf8]{inputenc} 
\usepackage[T1]{fontenc}
\usepackage{geometry} 
\geometry{a4paper} 
\usepackage{graphicx} 

%%% PACKAGES
\usepackage{booktabs} % for much better looking tables
\usepackage{array} % for better arrays (eg matrices) in maths
\usepackage{paralist} % very flexible & customisable lists (eg. enumerate/itemize, etc.)
\usepackage{verbatim} % adds environment for commenting out blocks of text & for better verbatim
\usepackage{subfig} % make it possible to include more than one captioned figure/table in a single float


%%% HEADERS & FOOTERS
\usepackage{fancyhdr} % This should be set AFTER setting up the page geometry
\pagestyle{fancy} % options: empty , plain , fancy
\renewcommand{\headrulewidth}{0pt} % customise the layout...
\lhead{}\chead{}\rhead{}
\lfoot{}\cfoot{\thepage}\rfoot{}

%%% SECTION TITLE APPEARANCE
\usepackage{sectsty}
\allsectionsfont{\sffamily\mdseries\upshape} % (See the fntguide.pdf for font help)


%%% ToC (table of contents) APPEARANCE
\usepackage[nottoc,notlof,notlot]{tocbibind} % Put the bibliography in the ToC
\usepackage[titles,subfigure]{tocloft} % Alter the style of the Table of Contents
\renewcommand{\cftsecfont}{\rmfamily\mdseries\upshape}
\renewcommand{\cftsecpagefont}{\rmfamily\mdseries\upshape} % No bold!
%%% END Article customization



%%% The "real" document content comes below...
\title{\textbf{CS498 Cloud Computing Applications:\\ Deutsche B\"orse Public Dataset}}
\author{Marjan Ahmed, Fan Yang, Dilruba Hawk and Wang Chun Wei }
\date{marjana2, fanyang3, dilruba2, wcwei2}

\begin{document}
\maketitle


\section{The Publicly Hosted Dataset}
The dataset we pick is the Deutsche B\"orse public dataset on AWS.
This dataset contains trading data (price and volume) in 1 minute intervals for every tradeable security listed on the Eurex and Xetra trading platforms located in Frankfurt, Germany\footnote{Data link: https://registry.opendata.aws/deutsche-boerse-pds/}.

\section{Research Project}
We intend to build a cloud hosted backtesting framework for algorithmic trading. 
Given that the dataset only contains price-volume data, we focus on testing only momentum based trading strategies. A simple web interface will allow users to contol and tweak momentum parameter settings, and see in-sample and out-of-sample trading performance (i.e., hypothetical profit and loss, trading costs, turnover, portfolio risk and Sharpe ratio). 
The objective is to develop optimal momentum strategies for trading German equities quickly via cloud computing.

\subsection{Intellectual Merit}
Seminal research by Jedadeesh and Titman (1993) show that forming portfolios based on past stock returns (6-12 months) yields positive returns over the next 3-12 months.
This behavior of price continuation has attributed to investor behavioral biases and investor under-reaction, and have been shown to be prevalent across countries (Griffin et al. 2003), industries (Moskowitz and Grinblatt, 1999) and asset classes (Miffre and Rallis, 2007).
 


\subsection{Broader Impact}
The project allows users to backtest and examine the efficacy of different momentum strategies in Germany.
This is a useful tool for practitioners in the asset management industry as momentum is a key investment style\footnote{Momentum ETFs are very popular on with investors: https://etfdb.com/etfs/investment-style/high-momentum/ }.
Furthermore, the project framework can be expanded for live trading (if there are live data feeds). 


\begin{thebibliography}{}
\bibitem{1}  Griffin, J.M., Ji, X. and S. Martin (2003) Momentum investing and  business cycle risk: Evidence from pole to pole,  \emph{Journal of Finance}, 58, 2515 - 2547
\bibitem{2} Jegadeesh, N., and S. Titman (1993) Returns to buying winners and selling losers: Implications for stock market efficiency, \emph{Journal of Finance}, 48, 65 - 91
\bibitem{3}  Miffre, J. and G. Rallis (2007) Momentum strategies in commodities markets,  \emph{Journal of Banking and Finance}, 31 (6), 1863 - 1886
\bibitem{4} Moskowitz, T.J. and M. Grinblatt (1999) Do industries explain momentum?, \emph{Journal of Finance}, 54, 1249 - 1290
 \end{thebibliography}

\end{document}
